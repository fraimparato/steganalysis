\documentclass[../../main.tex]{subfiles}

\begin{document}
\begin{abstract}

This paper provides a general view about steganalysis with hints to
steganography. 

Steganography is the art of hide information into a cover type (the mean of
transmission used to hide the message) \cite{steganography-definition}.
Steganalysis is the analysis used to find whether in a cover type there is an
hidden message.

We will cover the state of art of the most important and used methods of
steganalysis for different cover types:

\begin{itemize}
    \item images: stego-key search, difference image histogram method, closest
          color pair method, JPEG steganalysis
    \item audio: non compressed and compressed methods, phase coding, Mp3
          steganalysis
    \item text
    \item TCP/IP
\end{itemize}

Moreover, we will give describe the general types of analysis that one can
perform:
\begin{multicols}{2}
    \begin{itemize}
        \item statical
        \item visual
        \item spread spectrum
        \item signature
        \item transform domain
        \item universal
    \end{itemize}
\end{multicols}

At the end we will conclude explaining why steganography and steganalysis are
important, also joining those two subjects with a wider theme like security in
sending information.

\end{abstract}

\pagebreak
\end{document}