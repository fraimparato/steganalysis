\documentclass[../../main.tex]{subfiles}

\begin{document}

\begin{center}
    \center \textbf{Abstract}
\end{center}

\vspace{0.1cm}

The aim of this paper is providing a general view of steganalysis declined in its aspects
while giving insights on the steganographic tools used. 

Steganography is the art of hide information into a cover type (the mean of
transmission used to hide the message) \cite{steganography-definition}.
Steganalysis is the art of analysing the cover type in order to detect the presence and in some cases even extract hidden information.

We will cover the state of art of the most important and used methods in
steganalysis for different cover types: images (stego-key search, difference
image histogram method, closest color pair method, JPEG steganalysis), audio
(non compressed and compressed methods, phase coding, Mp3 steganalysis), text
and TCP/IP.

Moreover, we will give describe the general types of analysis that one can
perform: statical, visual, spread spectrum, signature, transform domain,
universal.

Finally we will highlight the deep relation between steganography and steganalysis and describe their 
role of protagonists in the field of information security.

\end{document}