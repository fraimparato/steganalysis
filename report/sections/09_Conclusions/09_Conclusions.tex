\documentclass[../../main.tex]{subfiles}

\begin{document}
\section{Conclusions}

Up to now we discussed what are steganography and steganalysis and how the
latter is performed. The reason why steganalysis is important is that nowadays
almost everyone uses a computer or a digital device, and the security of
informations is a primary object of attention. This attention led to a
developlment of studies on signal processing jointly with information security
services. Steganography is not the only way one can hide informations, there are
multiple options, each one of them with a specific target:

\begin{itemize}
    \item encryption: confidentiality
    \item watermarking: copyright protection
    \item steganography: privacy that can also prevent traffic analysis
\end{itemize}

We can also make an example to clarify the difference between encryption and
steganography: \emph{the prisoners' problem} \cite{prisoners-problem}, which
considers that two accomplices in a crime have been arrested and are in two
different rooms. The warden decides to let them communicate by exchanging
written messages at the condition of reading all of them, hoping to find some
information and because he fears that they could share an escape plan. Now, if
the two prisoners encrypted their messages the warden would have been able to
notice that they are communicating something ``strange'' and probably he will
not deliver the message to the other criminal; in the case in which the two
prisoners steganographed the message, the warden would not have been able to
find that there were hidden information.

Given this example, we can also clarify what \emph{steganalysis} is: it can be
seen as the process that the warden should do in order to find that in the
``normal'' message of the prisoners there were hidden information.

At this point we can easily understand that the steganography method is
``\emph{broken}'' simply when steganalysis is able to detect that in the cover
type there is a hidden message. In general we can also distinguish between

\begin{itemize}
    \item \emph{passive steganalysis} which simply detects the message without
          knowing anything else
    \item \emph{active steganalysis} which detects the message with some extra
          information such as the length of the message and/or its location
\end{itemize}

Moreover, when steganalysis is performed successfully, we can also update the
steganography method which has been broken by the steganalysis. This continuous
research in steganography and steganalysis is useful when it comes to protect
data. \cite{review-audio-steganalysis}

\end{document}