\documentclass[../../main.tex]{subfiles}

\begin{document}


\section{Introduction}

Steganography is the technique of hiding a message inside another message or a physical object.
The word \textit{steganography} comes from the Greek word \textit{steganographia}, which is the combination of 
\textit{steganós} meaning ``covered'' and \textit{-graphia} meaning ``writing''.
\\
The first testimonials of the use of steganography date back to 440 BC in Greece mentioned by Herodotus in his
\emph{Histories}: Histitateus sent a message to Aristagoras by writing a text message on the shaved head of one of his
servants and then waited till the hair of the servant had regrow to sent him to Aristagoras.
Moreover steganography has been used for centuries in different ways such as secret inks, morse code hidden inside
physical objects or encoded in eyes blinking (Jeremiah Denton, tortured prisoner-of-war in 1966 during the korean
war, encoded in this way an help message during a TV report) or microdots embedded in paper or in clothes used by
espionage agents during and after the World War II.
\\
In digital steganography a message or a file is concealed within another file; in particular, electronic communications
may contain a steganographic coding inside a information vehicle such as a document, a program or a media file (image,
audio or video).
Media files are ideal for hiding messages since due to their large size, the modification needed to encode a
steganographic coding cause a subtle change that is unlikely to notice for someone who is not looking for it.
This is one of the greatest differences of steganography with respect to cryptography: in cryptography the encrypted
messages are visible so they attract interest and it's more likely that they will be subject to some type of attacks to
be decoded.
\\
In this paper we will focus on digital steganography, the way in which it can be performed, examples of its application
in known cases and in particular the threats and opportunities regarding the cybersecurity implications of its use in
ditigal and communications systems.

\pagebreak


\end{document}