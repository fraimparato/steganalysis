\documentclass[../../main.tex]{subfiles}

\begin{document}

\section{Steganography}

As we already stated, steganography is used to hide a message in a
\emph{cover type}. The latter can be any sort of information acting as a cover.

In all cases, the message is \emph{embedded} into the cover type and becomes part of it. Differently from
Watermarking where the mark is simply overlayed to the original message.

\subsection{Methods}

Steganographic methods may be grouped under two different classifications.

The first classification concentrates on the cover. We distinguish between injective and generative steganography.
injecting steganography focuses on the methods implied in the injection of the \emph{stego message} in the cover type.
Generative steganography instead consists in the group of methods aimed at constructing a cover from the stego message.

The second classification is oriented on the mathematical representation of the cover types(treated as signals). Heere we distinguish between 
substitutive and constructive steganography.
Substitutive steganography targets the background noise in the signals. In this case the noise is modified to embed the message.
This usually can be achieved by exploiting the \emph{LSB}\footnote{Least Significant Bit} of each byte,
which is most susceptible to error, to hide the message in the signal matrix.

Several drawbacks arise, such as the limited size of the message to be inserted
to avoid amplified distorsions of the original signals, the resilience to
various degrees of compression and signals transformations, and lastly the
transparency of the message.
We leave for a further discussion the different available tradeoffs, selectively
mitigating some of the aforementioned drawbacks. 

Similarly, constructive steganography integrates a noise model into the message.
The main drawback is that it is difficult to produce, and by design fragile to
attacks\footnote{When referring to attacks to steganography we intend the steganalysis techniques involved to detect the message}.
\end{document}