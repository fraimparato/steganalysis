\documentclass[../../main.tex]{subfiles}

\begin{document}

\section{Steganography}

As we already stated, steganography is used to hide a message in a
\emph{cover type}. We can define a cover type as the mean used to communicate,
which can be a text file, an image, an audio recording, but also people, like in
the case of the servants that we have mentioned earlier, etc\dots

In all these cases, the message has been \emph{embedded} into the cover type.
Here we can make a distinction between steganography and watermarking since in
both cases a message is encoded into a communication's mean, but with different
purposes.

\subsection{Steganography}

Steganographic methods may be grouped under two different classifications.

For the first, we distinguish injective and generative steganography.
While injective steganography consists in finding an optimal way of injecting
the message for a given cover type, generative steganography aims at generating
an adequate container optimal for hiding a sought message.

As for the second classification, we mainly discriminate between substitutive
and constructive steganography.

Substitutive steganography targets the noise in signals, to substitute the
atmospheric noise with a secret message, whether it concerns audio or digital
image signals.
Most commonly, it may be achieved by replacing the LSB (Least Significant Bit),
which is most susceptible to error, with the hidden message in the matrix of
signals.
Several drawbacks arise, such as the limited size of the message to be inserted
to avoid amplified distorsions of the original signals, the resilience to
various degrees of compression and signals transformations, and lastly the
transparency of the message.
We leave for a further discussion the different available tradeoffs, selectively
mitigating some of the aforementioned drawbacks. 

Similarly, constructive steganography integrates a noise model.
The main drawback is that it is hard to produce, and by design fragile to
attacks.

\pagebreak

\end{document}